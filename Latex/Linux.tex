\documentclass[12pt,a4paper]{article}
\title{Linux Notes}
\author{Claude Lu}

\usepackage{xcolor}     
\usepackage{float}
\usepackage{graphicx}
\usepackage{enumerate}
\usepackage{enumitem} %允許自定義bullet
\usepackage{amsmath, amssymb, amsfonts} %basic package for math expressions.
\usepackage{listings}
\usepackage[top=1in, bottom=1in, left=1in, right=1in]{geometry}
%表格必用
\usepackage{multirow}
\usepackage{hhline, booktabs}
%enable hyperlink
\usepackage{hyperref}
\usepackage{array}

%enable lstlisting
\usepackage{listings}
\usepackage{verbatim}
\usepackage{framed}

%自定義顏色
\definecolor{rublue}{HTML}{0036A7}
\definecolor{rured}{HTML}{D62718}
\definecolor{vscodegreen}{HTML}{4EC9B0}
\definecolor{vscodeblack}{HTML}{1E1E1E}
%\definecolor{vscodeblack}{HTML}{4D4D4D}
\definecolor{vscodeyellow}{HTML}{DCDCAA}
\definecolor{vscodeblue}{HTML}{569CD6}
%自定義命令
\newcommand{\bluetexttt}[1]{\textcolor{blue}{\texttt{#1}}}
\newcommand{\redtexttt}[1]{\textcolor{red}{\texttt{#1}}}
\newcommand{\redtext}[1]{\textcolor{red}{#1}}

\newcommand{\rt}[1]{\textcolor{red}{#1}}
\newcommand{\bt}[1]{\textcolor{blue}{#1}}
\newcommand{\vsgt}[1]{\textcolor{vscodegreen}{#1}}
\newcommand{\vsft}[1]{\textcolor{vscodeyellow}{#1}}
\newcommand{\vsbt}[1]{\textcolor{vscodeblue}{#1}}

\newcommand{\struct}[1]{\colorbox{vscodeblack}{\vsgt{#1}}}
\newcommand{\function}[1]{\colorbox{vscodeblack}{\vsft{#1}}}
\newcommand{\macro}[1]{\colorbox{vscodeblack}{\vsbt{#1}}}


\newcommand{\bs}[1]{\boldsymbol{#1}}

\usepackage{verbatim}


\lstset{
	basicstyle=\ttfamily, %use typewriter font
	keywordstyle=\color{blue},%關鍵字顏色
	commentstyle=\color{teal}, %註釋顏色 
	stringstyle=\color{red}, % Strings in red
	tabsize=4, % Set tab size to 4 spaces
	showspaces=false, % Do not show spaces
	showstringspaces=true, % Do not show string spaces
	breaklines=true, %自動換行
	frame=single %在代碼外加上外框
}
\begin{document}
\maketitle
\section{Process}
\begin{enumerate}
	\item Run process in background by adding \bt{ \&}:
	\begin{footnotesize}
	\begin{verbatim}
	./your_executable &
	\end{verbatim}
	\item \textbf{UID}: user ID of the process owner, like \textit{root}, \textit{claude}
	\item \textbf{PID}: Process ID
	\item \textbf{PID}: Parent Process ID
	\item \textbf{C}: CPU utilization
	\item \textbf{STIME}: start time of the process
	\item \textbf{TTY}: Terminal associated with the process
	\item \textbf{TIME}: Total CPU time used
	\item \textbf{CMD}: Command that started the process
	\end{footnotesize}
\end{enumerate}
Common usage
\begin{footnotesize}
\begin{verbatim}
$ ps -ef
$ ps -fp <process ID>
$ ps -o pid,ppid,tty,time <process ID>
# Show call hierarchy
$ pstree -p <processID>
\end{verbatim}
\begin{enumerate}
	\item -p: required a process ID
	\item -o: option, including:
	\begin{enumerate}
		\item \textbf{pid}
		\item \textbf{ppid}
		\item \textbf{tty}
		\item \textbf{cputime}: CPU time used by the process
		\item \textbf{etime}: Elapsed time in MM:SS
		\item \textbf{stime}: start time of the process
		\item \textbf{args}: command with all its arguments
	\end{enumerate}
\end{enumerate}
\end{footnotesize}

\section{chmod}
Use in numeric mode
\begin{footnotesize}
\begin{verbatim}
# Open all permission
# Owner(user) | Group | Others
chmod 777 <file>
# r (read)=4
# w (write)=2
# x (execute)=1
\end{verbatim}
\end{footnotesize}

\section{date and time}
\begin{footnotesize}
\begin{verbatim}
$ date +%[OPTION]
OPTION
%F: %+4Y-%m-%d
%r: 12 hour
\end{verbatim}
\end{footnotesize}

\section{Package}
\subsection{apt}
\begin{footnotesize}
\begin{verbatim}
sudo apt update
# upgrade all installed packages to their lastest versions
sudo apt upgrade

# install a package
sudo apt install <package1> <package2> <package3>

# remove a package
sudo apt remove <package>

# removes both package and its configuration file
sudo apt purge <package>

# search for a package
apt search <package>

# show information about a package
apt show <package>

# list installed package
apt list --installed

# clean up unusded packages and dependencies
sudo apt autoremove

# clean package cache
sudo apt clean

# fix broken dependencies
sudo apt --fix-broken install 

# upgrade distribution (don't try easily)
sudo do-release-upgrade

# install local .deb package
sudo dpkg -i <package.deb>

# fix a package at certain package
sudo apt-mark hold <package>
sudo apt-mark unmode <package>
\end{verbatim}
\end{footnotesize}

\section{ssh}
\begin{enumerate}
	\item \texttt{ssh} is the \textit{client program} of the OepnSSH
	\item To set up an SSH connection, you need the remote host's \textit{public key}
	\item Public key cryptography have two keys:
	\begin{enumerate}
		\item public key: can encrypt a message but not decrypt it
		\item private: can decrypt a message from the public key
	\end{enumerate}
	\item the package name of ssh server on debian are \texttt{openssh-server}, and its systemd service name is \texttt{ssh}
	\item Check configurations and hosts keys in \texttt{/etc/ssh}:
	\begin{enumerate}
		\item \texttt{sshd\_config}: server configuration filename
		\item \texttt{ssh\_config}: client setup file
	\end{enumerate}
	\item OpenSSH Key Files: (rsa and dsa are just different algorithms)
	\begin{enumerate}
		\item ssh\_host\_rsa\_key
		\item ssh\_host\_rsa\_key.pub
		\item ssh\_host\_dsa\_key
		\item ssh\_host\_dsa\_key.pub		
	\end{enumerate}
	\item Use \texttt{fail2ban} to prevent malicious attacks from the internet
\end{enumerate}
ssh client usage
\begin{footnotesize}
\begin{verbatim}
# log in to a remote host
$ ssh remote_username@remote_host
\end{verbatim}
\end{footnotesize}
scp usage
\begin{footnotesize}
\begin{verbatim}
# transfer a file from local to remote
scp /path/to/local/file username@remote:/path/to/remote/directory
# transfer a directory from local to remote
scp -r /path/to/local/file username@remote:/path/to/remote/directory

# transfer a file from remote to local
scp username@remote:/path/to/remote/directory /path/to/local/file 
# transfer a directory from remote to local
scp -r username@remote:/path/to/remote/directory /path/to/local/file 
\end{verbatim}
\end{footnotesize}

\section{USB}

\subsection{99-com.rules}
Add symbolic according to physical port. First cd \textbf{/etc/udev/rules.d/99-com.rules}
\begin{footnotesize}
\begin{verbatim}
SUBSYSTEM=="tty", KERNELS="<your kernel>", SYMLINK+="<your name>"
\end{verbatim}
\end{footnotesize}

\begin{footnotesize}
\begin{verbatim}
# check attribute of a device
udevadm info --name=/dev/ttyACM* --attribute-walk
\end{verbatim}
\end{footnotesize}

\section{Shell Script}
\subsection{Rule of Thumb}
\begin{enumerate}
	\item All bash script shall shall start with \bt{\#!/bin/sh}
	\item \bt{\#!} Reads \textbf{sharp bang} or \textbf{Shebang}
	\item If you need python to do the work, use \bt{\#!/usr/bin/python} instead
\end{enumerate}

\subsection{Special Variables}
\begin{enumerate}
	\item \textbf{Individual Arguments}: \bt{\$1}, \bt{\$2}, representing the n-th argument of the bash script, you can think of it as the combination of \bt{argc} and \bt{argv}. One can utilize \bt{shift} command to increment the number of individual arguments by one.
	\item \textbf{Number of Arguments}: \bt{\$\#}
	\item \textbf{All Arguments}: \bt{\$@}
	\item \textbf{Script Name}: \bt{\$0}
	\item \textbf{Process ID}: \bt{\$\$}
	\item \textbf{Exit Code}: \bt{\$?} The exit code holds the \rt{last} command that shell executed
\end{enumerate}

\subsection{self-defined variables}
\begin{enumerate}
	\item \rt{No} spaces before and after the \textit{equal sign}
	\item Variables are case-sensitive, and should be in \rt{uppercase}
\end{enumerate}
\begin{footnotesize}
\begin{verbatim}
#!/bin/bash
# Define your variable
VARIABLE_NAME="VALUE
# Use your variable
echo "This is my variable ${VARIABLE_NAME}"
# assign the output of a command as variable
VARIABLE=$(<command>)
VARIABLE=$`<command>`
\end{verbatim}
\end{footnotesize}

\subsection{list and arrays}
\begin{footnotesize}
\begin{verbatim}
# create an array
declare -a ARRAY

# add element to array
ARRAY+=("element1")
\end{verbatim}
\end{footnotesize}

\subsection{Conditionals}
\begin{footnotesize}
\begin{verbatim}
# establish a condition expression between brackets
[ condition-to-test-for ]
\end{verbatim}
\end{footnotesize}
File operators
\begin{enumerate}
	\item -d FILE if file is a directory
	\item -e FILE if file exists
	\item -f FILE if file exists and is a regular file
	\item -r FILE if file is readable by you
	\item -s FILE if file exists and is not empty
	\item -w FILE if file is writable by you
	\item -x FILE if file is executable by you 
\end{enumerate}
String operators:
\begin{enumerate}
	\item -z STRING if string is empty
	\item -n STRING is string is not empty
	\item STRING1 = STRING2 if strings are equal
	\item STRING1 != STRING2 if strings are not equal
\end{enumerate}
Arithmeitc operators:
\begin{enumerate}
	\item arg1 -eq arg2 : arg1 = arg2
	\item arg1 -ne arg2 : arg1 != arg2
	\item arg1 -lt arg2 : arg1 \textless arg2
	\item arg1 -le arg2 : arg1 \textless= arg2
	\item arg1 -gt arg2 : arg1 \textgreater arg2
	\item arg1 -ge arg2 : arg1 \textgreater= arg2
\end{enumerate}
\subsection{if statement}
\begin{footnotesize}
\begin{verbatim}
# Must be space between conditional and if
# Must have space after left bracket and before right bracket
# Must have space before and after equal sign when used for conditionals
# use && for AND, || for or
if [ condition-true ]
then
  command 1
  command 2
elif [ condition-true ]
then
  command 3
  command 4
else
  command 5
  command 6
fi
\end{verbatim}
\end{footnotesize}
One can directly utilize the exit code of a command as the condition of if statement:
\begin{footnotesize}
\begin{verbatim}
if <command>; then
// your code
fi
\end{verbatim}
\end{footnotesize}
\subsection{for loop}
\begin{footnotesize}
\begin{verbatim}
# ITEM should be seperated by space
for VARIABLE_NAME in ITEM1 ITEM2 ITEM3
do
  command 1
  command 2
done
# One can store list of items in variable, then iterate
ITEMS="ITEM1 ITEM2 ITEM3"
for ITEM in ${ITMES}
do
  command 1
  command 2
done
\end{verbatim}
\end{footnotesize}
An array-based for-loop
\begin{footnotesize}
\begin{verbatim}
array=(itme1 item2 item3)

for item in "${array[@]}"; do
// your code
done
\end{verbatim}
\end{footnotesize}
A wildcard for loop
\begin{footnotesize}
\begin{verbatim}
# Search all .pattern file in the directory same as the bashscript
for FILE in *.pattern
do
// code
done
\end{verbatim}
\end{footnotesize}
Use pipe \texttt{|} to represent ``or'' in case 
\subsection{case statements}
\begin{footnotesize}
\begin{verbatim}
read -p "Enter y or n: " ANSWER
case "$ANSWER" in
  [yY]|[yY][eE][sS])
    echo "You enter yes"
    ;;
  [nN]|[nN][oO])
    echo "You enter no"
    ;;
  *)
    echo "Invalid answer"
    ;;
esac
\end{verbatim}
\end{footnotesize}

\subsection{read}
\begin{footnotesize}
	\begin{verbatim}
		read -p "ENTER THE INPUT: " INPUT
	\end{verbatim}
\end{footnotesize}
A common design pattern
\begin{footnotesize}
\begin{verbatim}
read -p "Enter y or n: " ANSWER
case "$ANSWER" in
	[yY][]
\end{verbatim}
\end{footnotesize}

\subsection{Logical Operator}
\begin{enumerate}
	\item the first second command will execute \rt{if and only if} the first one exit with 0
	\begin{verbatim}
	command1 && command2
	\end{verbatim}
	\item the second command will execute if and only if the first one \rt{failed}, in other word, if the first command succeed, the second one won't execute
	\begin{verbatim}
	command1 || command2
	\end{verbatim}
	\item two commands will execute no matter what
	\begin{verbatim}
	command1 ; command2
	\end{verbatim}
\end{enumerate}

\subsection{exit}
\begin{enumerate}
	\item use \textbf{exit} command with a number from 0 to 255
	\item If no exit code is specified, the previously executed command is used as the exit status
\end{enumerate}

\subsection{function}
\begin{footnotesize}
\begin{verbatim}
# create a function
# Method 1
function funcion-name(){
  # Code
}
# Method 2
function-name(){
  # Code
}

#Passing arguments
function-name arg1 arg2 arg3
\end{verbatim}
\end{footnotesize}

\subsection{wildcard}
\begin{enumerate}
	\item \textbf{wildcard} is a character set used to represent one or more characters in file or directory names.
	\item \texttt{*}: asterisk, zero or more character
	\item \texttt{?}: question mark, one character
	\item \texttt{[]}: square brackets, any single character within the brackets.
	\item \texttt{[!...]}: negated square brackets, matches any character \textbf{except} those in the character
	\item \textbf{character class}: \texttt{ca[nt]*} will match can, cat, canbus, or catch
	\item \textbf{ranges}: \texttt{[a-g]} match all characters from "a" to "g", [3-6] match the numbers 3, 4, 5
	\item \textbf{named character classes}:
	\begin{enumerate}
		\item \texttt{[:alpha:]} Matches lowercase and uppercase letters
		\item \texttt{[:alnum:]} Matches alpha and digits.
		\item \texttt{[:digit:]} Represents numbers in decimal
		\item \texttt{[:upper:]} Matches uppercase letters
		\item \texttt{[:lower:]} Matches lowercase letters
		\item \texttt{[:space:]} Matches whitespace, such as spaces, tabs and newline characters
	\end{enumerate}		
	\item \textbf{Matching Wildcard Patterns}: to escape from wildcard patterns, use \textbf{backslash}
\end{enumerate}

\section{Embedded Linux}
Four element of embedded Linux:
\begin{enumerate}
	\item Toolchain: the compiler and other tools needed to create code for the target device
	\item Bootloader: the program that initializes the board and loads the Linux kernel
	\item Kernel: managing system resources and interfacing with hardware
	\item Root filesystem: libraries and programs that are run once kernel has completed initialization
\end{enumerate}

\subsection{Toolchain}
\begin{enumerate}
	\item Toolchain comprising of the followings:
	\begin{enumerate}
		\item compiler
		\item linker
		\item runtime libraries
	\end{enumerate}
	\item \bt{bootloader}, \bt{kernel} and \bt{root filesystem} are compiled by toolchain
	\item GNU tool chain is composed of three things:
	\begin{enumerate}
		\item \textbf{Binutils}
		\item \textbf{GNU Compiler Collection} (GCC)
		\item \textbf{C library}: a standardized application program interface (API) based on POSIX specification.
	\end{enumerate}
	\item \textbf{headers} should be from, or older than the kernel your using.
	\item \textbf{GNU Debugger} (GDB) is usually considered a part of the tool chain.
	\item toolchain can be categorized as below:
	\begin{enumerate}
		\item Native: this toolchain runs on the same type of system as the program it generates.
		\item Cross: this toolchain runs on a different type of system than the target
	\end{enumerate}
	\item to build toolchain, must consider the following things:
	\begin{enumerate}
		\item CPU architecture
		\item big or little endian
		\item floating point support
	\end{enumerate}
	\item application binary interface (ABI): how different pieces of compiled code (binaries) work together. For example, ARM use \textbf{Extended Application Binary Interface} (EABI)
	\item The programming interface to Unix operating system is defined in the \bt{C language}, which is defined by \textbf{POSIX}
	\item \textbf{C Library} is the \textit{implementation} of Portable Operating System Interface (POSIX):
	\begin{enumerate}
		\item \textbf{glibc}: use this !
		\item \textbf{musl libc}: use when storage less than 32 MiB
		\item \textbf{uClibc-ng}
		\item \textbf{eglibc}
	\end{enumerate}
	\item All the applications need to communicate with Linux kernel through the \bt{C library}
\end{enumerate}
	
\end{document}